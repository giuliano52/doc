\documentclass[]{article}
\usepackage{lmodern}
\usepackage{amssymb,amsmath}
\usepackage{ifxetex,ifluatex}
\usepackage{fixltx2e} % provides \textsubscript
\ifnum 0\ifxetex 1\fi\ifluatex 1\fi=0 % if pdftex
  \usepackage[T1]{fontenc}
  \usepackage[utf8]{inputenc}
\else % if luatex or xelatex
  \ifxetex
    \usepackage{mathspec}
    \usepackage{xltxtra,xunicode}
  \else
    \usepackage{fontspec}
  \fi
  \defaultfontfeatures{Mapping=tex-text,Scale=MatchLowercase}
  \newcommand{\euro}{€}
\fi
% use upquote if available, for straight quotes in verbatim environments
\IfFileExists{upquote.sty}{\usepackage{upquote}}{}
% use microtype if available
\IfFileExists{microtype.sty}{\usepackage{microtype}}{}
\usepackage[top=1cm, bottom=1.5cm, left=1cm, right=1cm]{geometry}
\usepackage{color}
\usepackage{fancyvrb}
\newcommand{\VerbBar}{|}
\newcommand{\VERB}{\Verb[commandchars=\\\{\}]}
\DefineVerbatimEnvironment{Highlighting}{Verbatim}{commandchars=\\\{\}}
% Add ',fontsize=\small' for more characters per line
\newenvironment{Shaded}{}{}
\newcommand{\KeywordTok}[1]{\textcolor[rgb]{0.00,0.44,0.13}{\textbf{{#1}}}}
\newcommand{\DataTypeTok}[1]{\textcolor[rgb]{0.56,0.13,0.00}{{#1}}}
\newcommand{\DecValTok}[1]{\textcolor[rgb]{0.25,0.63,0.44}{{#1}}}
\newcommand{\BaseNTok}[1]{\textcolor[rgb]{0.25,0.63,0.44}{{#1}}}
\newcommand{\FloatTok}[1]{\textcolor[rgb]{0.25,0.63,0.44}{{#1}}}
\newcommand{\CharTok}[1]{\textcolor[rgb]{0.25,0.44,0.63}{{#1}}}
\newcommand{\StringTok}[1]{\textcolor[rgb]{0.25,0.44,0.63}{{#1}}}
\newcommand{\CommentTok}[1]{\textcolor[rgb]{0.38,0.63,0.69}{\textit{{#1}}}}
\newcommand{\OtherTok}[1]{\textcolor[rgb]{0.00,0.44,0.13}{{#1}}}
\newcommand{\AlertTok}[1]{\textcolor[rgb]{1.00,0.00,0.00}{\textbf{{#1}}}}
\newcommand{\FunctionTok}[1]{\textcolor[rgb]{0.02,0.16,0.49}{{#1}}}
\newcommand{\RegionMarkerTok}[1]{{#1}}
\newcommand{\ErrorTok}[1]{\textcolor[rgb]{1.00,0.00,0.00}{\textbf{{#1}}}}
\newcommand{\NormalTok}[1]{{#1}}
\ifxetex
  \usepackage[setpagesize=false, % page size defined by xetex
              unicode=false, % unicode breaks when used with xetex
              xetex]{hyperref}
\else
  \usepackage[unicode=true]{hyperref}
\fi
\hypersetup{breaklinks=true,
            bookmarks=true,
            pdfauthor={Giuliano Dedda},
            pdftitle={Bash},
            colorlinks=true,
            citecolor=blue,
            urlcolor=blue,
            linkcolor=magenta,
            pdfborder={0 0 0}}
\urlstyle{same}  % don't use monospace font for urls
\setlength{\parindent}{0pt}
\setlength{\parskip}{6pt plus 2pt minus 1pt}
\setlength{\emergencystretch}{3em}  % prevent overfull lines
\setcounter{secnumdepth}{0}

\title{Bash}
\author{Giuliano Dedda}
\date{10/07/2014}

\begin{document}
\maketitle

\section{Esempi}\label{esempi}

\subsection{WalkTree}\label{walktree}

Mostra ricorsivamente le directory

\begin{Shaded}
\begin{Highlighting}[]
\KeywordTok{function}\FunctionTok{ walk_tree} \KeywordTok{\{}
      \KeywordTok{echo} \StringTok{"Directory: }\OtherTok{$1}\StringTok{"}
      \KeywordTok{local} \OtherTok{directory=}\StringTok{"}\OtherTok{$1}\StringTok{"}
      \KeywordTok{local} \OtherTok{i}
      \KeywordTok{for} \KeywordTok{i} \NormalTok{in }\StringTok{"}\OtherTok{$directory}\StringTok{"}\NormalTok{/*}\KeywordTok{;} 
      \KeywordTok{do}
      \KeywordTok{echo} \StringTok{"File: }\OtherTok{$i}\StringTok{"}
        \KeywordTok{if [} \StringTok{"}\OtherTok{$i}\StringTok{"} \OtherTok{=} \NormalTok{. }\OtherTok{-o} \StringTok{"}\OtherTok{$i}\StringTok{"} \OtherTok{=} \NormalTok{..}\KeywordTok{ ]}\NormalTok{; }\KeywordTok{then} 
            \KeywordTok{continue}
        \KeywordTok{elif [} \OtherTok{-d} \StringTok{"}\OtherTok{$i}\StringTok{"}\KeywordTok{ ]}\NormalTok{; }\KeywordTok{then}  \CommentTok{# Process directory and / or walk-down into directory}
            \CommentTok{# add command here to process all files in directory (i.e. ls -l "$i/"*)}
            \KeywordTok{walk_tree} \StringTok{"}\OtherTok{$i}\StringTok{"}      \CommentTok{# DO NOT COMMENT OUT THIS LINE!!}
        \KeywordTok{else}
            \KeywordTok{continue}    \CommentTok{# replace continue to process individual file (i.e. echo "$i")}
        \KeywordTok{fi}
      \KeywordTok{done}
\KeywordTok{\}}

\KeywordTok{walk_tree} \OtherTok{$HOME}
\end{Highlighting}
\end{Shaded}

\subsection{Ciclo For}\label{ciclo-for}

Esegui comando per ogni file

\begin{Shaded}
\begin{Highlighting}[]
\KeywordTok{for} \KeywordTok{f} \NormalTok{in dir_to_scan/* }
\KeywordTok{do} 
   \KeywordTok{echo} \OtherTok{$f} 
\KeywordTok{done}
\end{Highlighting}
\end{Shaded}

o in una sola riga

\begin{verbatim}
for f in dir_to_scan/* ; do echo $f ; done
\end{verbatim}

\end{document}
