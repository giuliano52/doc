\documentclass[]{article}
\usepackage{lmodern}
\usepackage{amssymb,amsmath}
\usepackage{ifxetex,ifluatex}
\usepackage{fixltx2e} % provides \textsubscript
\ifnum 0\ifxetex 1\fi\ifluatex 1\fi=0 % if pdftex
  \usepackage[T1]{fontenc}
  \usepackage[utf8]{inputenc}
\else % if luatex or xelatex
  \ifxetex
    \usepackage{mathspec}
    \usepackage{xltxtra,xunicode}
  \else
    \usepackage{fontspec}
  \fi
  \defaultfontfeatures{Mapping=tex-text,Scale=MatchLowercase}
  \newcommand{\euro}{€}
\fi
% use upquote if available, for straight quotes in verbatim environments
\IfFileExists{upquote.sty}{\usepackage{upquote}}{}
% use microtype if available
\IfFileExists{microtype.sty}{\usepackage{microtype}}{}
\usepackage[top=1cm, bottom=1.5cm, left=1cm, right=1cm]{geometry}
\usepackage{color}
\usepackage{fancyvrb}
\newcommand{\VerbBar}{|}
\newcommand{\VERB}{\Verb[commandchars=\\\{\}]}
\DefineVerbatimEnvironment{Highlighting}{Verbatim}{commandchars=\\\{\}}
% Add ',fontsize=\small' for more characters per line
\newenvironment{Shaded}{}{}
\newcommand{\KeywordTok}[1]{\textcolor[rgb]{0.00,0.44,0.13}{\textbf{{#1}}}}
\newcommand{\DataTypeTok}[1]{\textcolor[rgb]{0.56,0.13,0.00}{{#1}}}
\newcommand{\DecValTok}[1]{\textcolor[rgb]{0.25,0.63,0.44}{{#1}}}
\newcommand{\BaseNTok}[1]{\textcolor[rgb]{0.25,0.63,0.44}{{#1}}}
\newcommand{\FloatTok}[1]{\textcolor[rgb]{0.25,0.63,0.44}{{#1}}}
\newcommand{\CharTok}[1]{\textcolor[rgb]{0.25,0.44,0.63}{{#1}}}
\newcommand{\StringTok}[1]{\textcolor[rgb]{0.25,0.44,0.63}{{#1}}}
\newcommand{\CommentTok}[1]{\textcolor[rgb]{0.38,0.63,0.69}{\textit{{#1}}}}
\newcommand{\OtherTok}[1]{\textcolor[rgb]{0.00,0.44,0.13}{{#1}}}
\newcommand{\AlertTok}[1]{\textcolor[rgb]{1.00,0.00,0.00}{\textbf{{#1}}}}
\newcommand{\FunctionTok}[1]{\textcolor[rgb]{0.02,0.16,0.49}{{#1}}}
\newcommand{\RegionMarkerTok}[1]{{#1}}
\newcommand{\ErrorTok}[1]{\textcolor[rgb]{1.00,0.00,0.00}{\textbf{{#1}}}}
\newcommand{\NormalTok}[1]{{#1}}
\ifxetex
  \usepackage[setpagesize=false, % page size defined by xetex
              unicode=false, % unicode breaks when used with xetex
              xetex]{hyperref}
\else
  \usepackage[unicode=true]{hyperref}
\fi
\hypersetup{breaklinks=true,
            bookmarks=true,
            pdfauthor={Giuliano Dedda},
            pdftitle={PHP},
            colorlinks=true,
            citecolor=blue,
            urlcolor=blue,
            linkcolor=magenta,
            pdfborder={0 0 0}}
\urlstyle{same}  % don't use monospace font for urls
\setlength{\parindent}{0pt}
\setlength{\parskip}{6pt plus 2pt minus 1pt}
\setlength{\emergencystretch}{3em}  % prevent overfull lines
\setcounter{secnumdepth}{0}

\title{PHP}
\author{Giuliano Dedda}
\date{21/07/2014}

\begin{document}
\maketitle

\section{Array}\label{array}

\begin{Shaded}
\begin{Highlighting}[]
\KeywordTok{$a}\NormalTok{=}\FunctionTok{array}\OtherTok{(}\StringTok{"1"}\OtherTok{,}\StringTok{"2"}\OtherTok{,}\StringTok{"3"}\OtherTok{);}
\KeywordTok{foreach}\OtherTok{(}\KeywordTok{$a} \KeywordTok{as} \KeywordTok{$valore}\OtherTok{)} \NormalTok{\{}
    \FunctionTok{echo} \KeywordTok{$valore}\OtherTok{;}
    \NormalTok{\}}
\end{Highlighting}
\end{Shaded}

Mostra tutto il contenuto di un array

\begin{verbatim}
var_dump        
var_export
print_r     
\end{verbatim}

\section{Acceleratore}\label{acceleratore}

\begin{verbatim}
aptitude install php-apc
\end{verbatim}

\section{Funzioni utili}\label{funzioni-utili}

Per provare il PHP

\begin{verbatim}
phpinfo();
\end{verbatim}

Per Passare i parametri con spazi ed altro uso le funzioni

\begin{verbatim}
base64_encode / base64_decode
\end{verbatim}

Per eliminare i tag HTML

\begin{verbatim}
strip_tags($string) 
\end{verbatim}

Per trasformare una stringa con `è' `\textgreater{}' `\&' \ldots{} in
una con i caratteri HTML \texttt{\&egrave;} \ldots{}

\begin{verbatim}
htmlentities($string)
\end{verbatim}

Come sopra solo che trasforma esclusivamente: \textless{} \textgreater{}
\& "

\begin{verbatim}
htmlspecialchar($string)
\end{verbatim}

Per verificare se una stringa contiene un valore

\begin{Shaded}
\begin{Highlighting}[]
\ErrorTok{ereg}\OtherTok{(}\StringTok{"ab"}\OtherTok{,}\KeywordTok{$string}\OtherTok{)}
\KeywordTok{true} \NormalTok{se }\KeywordTok{$string} \NormalTok{contiene ab}
\end{Highlighting}
\end{Shaded}

\section{Files}\label{files}

per analizzare un file:

\begin{Shaded}
\begin{Highlighting}[]
    \KeywordTok{$lines} \NormalTok{= }\FunctionTok{file} \OtherTok{(}\StringTok{"nomefile.txt"}\OtherTok{);}
    \KeywordTok{foreach} \OtherTok{(}\KeywordTok{$lines} \KeywordTok{as} \KeywordTok{$line_num} \NormalTok{=> }\KeywordTok{$line}\OtherTok{)} \NormalTok{\{}
        \FunctionTok{echo} \StringTok{"Line #<b>}\KeywordTok{\{$line_num\}}\StringTok{</b> : "} \NormalTok{. }\FunctionTok{htmlspecialchars}\OtherTok{(}\KeywordTok{$line}\OtherTok{)} \NormalTok{. }\StringTok{"<br>}\KeywordTok{\textbackslash{}n}\StringTok{"}\OtherTok{;}
    \NormalTok{\}}
\end{Highlighting}
\end{Shaded}

Elenca tutti i file della directory ed elimina . e ..

\begin{Shaded}
\begin{Highlighting}[]
\KeywordTok{$sDir} \NormalTok{= }\StringTok{"/tmp/"}\OtherTok{;}
\CommentTok{// Open a known directory, and proceed to read its contents}
\KeywordTok{if} \OtherTok{(}\FunctionTok{is_dir}\OtherTok{(}\KeywordTok{$sDir}\OtherTok{))} \NormalTok{\{}
    \KeywordTok{if} \OtherTok{(}\KeywordTok{$dh} \NormalTok{= }\FunctionTok{opendir}\OtherTok{(}\KeywordTok{$sDir}\OtherTok{))} \NormalTok{\{}
        \KeywordTok{while} \OtherTok{((}\KeywordTok{$file} \NormalTok{= }\FunctionTok{readdir}\OtherTok{(}\KeywordTok{$dh}\OtherTok{))} \NormalTok{!== }\KeywordTok{false}\OtherTok{)} \NormalTok{\{}
            \KeywordTok{if} \OtherTok{(}\KeywordTok{$file} \NormalTok{!= }\StringTok{"."} \NormalTok{&& }\KeywordTok{$file} \NormalTok{!= }\StringTok{".."}\OtherTok{)} \NormalTok{\{}
                \FunctionTok{echo} \StringTok{"filename: <b>}\KeywordTok{$file}\StringTok{</b> <br />}\KeywordTok{\textbackslash{}n}\StringTok{"}\OtherTok{;}
            \NormalTok{\} }
        \NormalTok{\}}
        \FunctionTok{closedir}\OtherTok{(}\KeywordTok{$dh}\OtherTok{);}
    \NormalTok{\}}
\NormalTok{\}}
\end{Highlighting}
\end{Shaded}

oppure per mostrare tutti i file si può usare la funzione \emph{glob}

\begin{Shaded}
\begin{Highlighting}[]
\KeywordTok{foreach} \OtherTok{(}\FunctionTok{glob}\OtherTok{(}\StringTok{"felect/*.html"}\OtherTok{)} \KeywordTok{as} \KeywordTok{$filename}\OtherTok{)}
\NormalTok{\{}
    \FunctionTok{echo} \StringTok{"}\KeywordTok{$filename\textbackslash{}n}\StringTok{"}\OtherTok{;}
    \CommentTok{//$file = file_get_contents($filename);}
    \CommentTok{//file_put_contents($filename, preg_replace("/regexhere/","replacement",$file));}
\NormalTok{\}}
\end{Highlighting}
\end{Shaded}

Mostra i dettagli di un file si può usare \emph{pathinfo}

\begin{Shaded}
\begin{Highlighting}[]
\KeywordTok{$path_parts} \NormalTok{= }\FunctionTok{pathinfo}\OtherTok{(}\StringTok{'/www/htdocs/index.html'}\OtherTok{);}
\KeywordTok{$path_parts}\OtherTok{[}\StringTok{'dirname'}\OtherTok{]}      \NormalTok{->  /www/htdocs}
\KeywordTok{$path_parts}\OtherTok{[}\StringTok{'basename'}\OtherTok{]}     \NormalTok{->  index.html}
\KeywordTok{$path_parts}\OtherTok{[}\StringTok{'extension'}\OtherTok{]}    \NormalTok{->  html}
\KeywordTok{$path_parts}\OtherTok{[}\StringTok{'filename'}\OtherTok{]}     \NormalTok{->  index}
\end{Highlighting}
\end{Shaded}

\section{Linea di comando}\label{linea-di-comando}

$argv è l'array con la linea di comando ($argv{[}0{]} è il nome dello
script) getopt() è la funzione che analizza le singole ozioni e
restituisce un array

\begin{Shaded}
\begin{Highlighting}[]
\CommentTok{// parse the command line ($GLOBALS['argv'])}
\KeywordTok{$options} \NormalTok{= }\FunctionTok{getopt}\OtherTok{(}\StringTok{"f:"}\OtherTok{);}  \CommentTok{// restituisce l'opzione -f}
\end{Highlighting}
\end{Shaded}

\section{Sqlite}\label{sqlite}

\begin{Shaded}
\begin{Highlighting}[]
\KeywordTok{if} \OtherTok{(}\KeywordTok{$db} \NormalTok{= }\FunctionTok{sqlite_open}\OtherTok{(}\StringTok{"dati.sqlite"}\OtherTok{,} \DecValTok{0666}\OtherTok{,} \KeywordTok{$sqliteerror}\OtherTok{))} \NormalTok{\{}
\KeywordTok{$query} \NormalTok{= }\KeywordTok{<<<EOD}
\StringTok{CREATE TABLE patch (}
\StringTok{    computer varchar(255), }
\StringTok{    bulletin varchar(255),}
\StringTok{    criticita varchar(30)}
\StringTok{    );}
\KeywordTok{EOD;}

    \FunctionTok{sqlite_query}\OtherTok{(}\KeywordTok{$db}\OtherTok{,} \KeywordTok{$query}\OtherTok{);}
\NormalTok{\} }\KeywordTok{else} \NormalTok{\{}
   \FunctionTok{die}\OtherTok{(}\KeywordTok{$sqliteerror}\OtherTok{);}
\NormalTok{\}}

\KeywordTok{$result} \NormalTok{= }\FunctionTok{sqlite_query}\OtherTok{(}\KeywordTok{$db}\OtherTok{,} \StringTok{"select * FROM patch "}\OtherTok{);}
\KeywordTok{while} \OtherTok{(}\KeywordTok{$row} \NormalTok{= }\FunctionTok{sqlite_fetch_array}\OtherTok{(}\KeywordTok{$result}\OtherTok{))} \NormalTok{\{}
    \CommentTok{// print_r($row);}
    \KeywordTok{$linea} \NormalTok{= }\KeywordTok{$row}\OtherTok{[}\StringTok{'computer'}\OtherTok{]} \NormalTok{. }\StringTok{"---"} \NormalTok{. }\KeywordTok{$row}\OtherTok{[}\StringTok{'bulletin'}\OtherTok{]} \NormalTok{. }\StringTok{"---"} \NormalTok{. }\KeywordTok{$row}\OtherTok{[}\StringTok{'criticita'}\OtherTok{]}\NormalTok{.}\StringTok{"+++"} \OtherTok{;}
    \FunctionTok{echo} \StringTok{"}\KeywordTok{$linea\textbackslash{}n}\StringTok{"}\OtherTok{;}
\NormalTok{\}}
\end{Highlighting}
\end{Shaded}

\section{Esecuzione di comandi di sitema con permessi
elevati}\label{esecuzione-di-comandi-di-sitema-con-permessi-elevati}

Se fosse necessario di eseguire comandi con permessi più elevati bisogna
lanciare il comando

\begin{verbatim}
sudo visudo 
\end{verbatim}

e poi aggiungere :

\begin{verbatim}
www-data ALL=NOPASSWD: /sbin/iptables, /usr/bin/du
\end{verbatim}

Per ad esempio permettere iptable e du

oppure:

\begin{verbatim}
www-data ALL=NOPASSWD: ALL
\end{verbatim}

per tutti i comandi. Nello script php si puù ora lanciare il comando:

\begin{verbatim}
exec ("sudo iptables -P FORWARD ACCEPT");
\end{verbatim}

\section{Upload}\label{upload}

Bisogna impostare i seguenti limiti nel file php.ini se si vuole fare
l'upload di file di grosse dimensioni

\begin{verbatim}
upload_max_filesize 20M
post_max_size 20M
max_execution_time 200
max_input_time 200
\end{verbatim}

Attenzione si può anche impostare 1G ma non 2G (va oltre il limite
consentito)

index.php

\begin{Shaded}
\begin{Highlighting}[]
\NormalTok{<html>}
\NormalTok{<body>}

\NormalTok{<form method=}\StringTok{"post"} \NormalTok{action=}\StringTok{"upload.php"} \NormalTok{enctype=}\StringTok{"multipart/form-data"}\NormalTok{>}
    \NormalTok{<input type=}\StringTok{"file"} \NormalTok{name=}\StringTok{"miofile"}\NormalTok{>}
    \NormalTok{<input type=}\StringTok{"submit"} \NormalTok{value=}\StringTok{"Upload"}\NormalTok{>}
\NormalTok{</form>}

\NormalTok{</body>}
\NormalTok{</html>}
\end{Highlighting}
\end{Shaded}

upload.php

\begin{Shaded}
\begin{Highlighting}[]
\NormalTok{<html>}
\NormalTok{<body>}
\NormalTok{<}\OtherTok{?}\KeywordTok{PHP}
    \CommentTok{// RECUPERO I PARAMETRI DA PASSARE ALLA FUNZIONE PREDEFINITA PER L'UPLOAD}
    \KeywordTok{$cartella} \NormalTok{= }\StringTok{'/mnt/sda3/upload/'}\OtherTok{;}
    \KeywordTok{$percorso} \NormalTok{= }\KeywordTok{$_FILES}\OtherTok{[}\StringTok{'miofile'}\OtherTok{][}\StringTok{'tmp_name'}\OtherTok{];}
    \KeywordTok{$nome} \NormalTok{= }\KeywordTok{$_FILES}\OtherTok{[}\StringTok{'miofile'}\OtherTok{][}\StringTok{'name'}\OtherTok{];}
    \CommentTok{// ESEGUO L'UPLOAD CONTROLLANDO L'ESITO}
    \KeywordTok{if} \OtherTok{(}\FunctionTok{move_uploaded_file}\OtherTok{(}\KeywordTok{$percorso}\OtherTok{,} \KeywordTok{$cartella} \NormalTok{. }\KeywordTok{$nome}\OtherTok{))}
    \NormalTok{\{}
        \FunctionTok{print} \StringTok{"Upload eseguito con successo"}\OtherTok{;}
    \NormalTok{\}}
    \KeywordTok{else}
    \NormalTok{\{}
        \FunctionTok{print} \StringTok{"Si sono verificati dei problemi durante l'Upload"}\OtherTok{;}
    \NormalTok{\}}
\KeywordTok{?>}
\NormalTok{</body>}
\NormalTok{</html>}
\end{Highlighting}
\end{Shaded}

\section{Mail}\label{mail}

Si può usare il comando mail:

\begin{verbatim}
mail( $destinatario,$oggetto , $messaggio, $intestazioni );
\end{verbatim}

se il mittente è nella forma:

\begin{verbatim}
$intestazioni = "From: $from\nReply-To: $from\nContent-Type: text/html";
\end{verbatim}

la mail diventa di tipo HTML

esempio più completo:

\begin{Shaded}
\begin{Highlighting}[]
\KeywordTok{$to} \NormalTok{= }\StringTok{"viralpatel.net@gmail.com"}\OtherTok{;}
\KeywordTok{$subject} \NormalTok{= }\StringTok{"VIRALPATEL.net"}\OtherTok{;}
\KeywordTok{$body} \NormalTok{= }\StringTok{"Body of your message here you can use HTML too. e.g. <br> <b> Bold </b>"}\OtherTok{;}
\KeywordTok{$headers} \NormalTok{= }\StringTok{"From: Peter}\KeywordTok{\textbackslash{}r\textbackslash{}n}\StringTok{"}\OtherTok{;}
\KeywordTok{$headers} \NormalTok{.= }\StringTok{"Reply-To: info@yoursite.com}\KeywordTok{\textbackslash{}r\textbackslash{}n}\StringTok{"}\OtherTok{;}
\KeywordTok{$headers} \NormalTok{.= }\StringTok{"Return-Path: info@yoursite.com}\KeywordTok{\textbackslash{}r\textbackslash{}n}\StringTok{"}\OtherTok{;}
\KeywordTok{$headers} \NormalTok{.= }\StringTok{"X-Mailer: PHP5}\KeywordTok{\textbackslash{}n}\StringTok{"}\OtherTok{;}
\KeywordTok{$headers} \NormalTok{.= }\StringTok{'MIME-Version: 1.0'} \NormalTok{. }\StringTok{"}\KeywordTok{\textbackslash{}n}\StringTok{"}\OtherTok{;}
\KeywordTok{$headers} \NormalTok{.= }\StringTok{'Content-type: text/html; charset=iso-8859-1'} \NormalTok{. }\StringTok{"}\KeywordTok{\textbackslash{}r\textbackslash{}n}\StringTok{"}\OtherTok{;}
\FunctionTok{mail}\OtherTok{(}\KeywordTok{$to}\OtherTok{,}\KeywordTok{$subject}\OtherTok{,}\KeywordTok{$body}\OtherTok{,}\KeywordTok{$headers}\OtherTok{);}
\end{Highlighting}
\end{Shaded}

Validazione email:

\begin{Shaded}
\begin{Highlighting}[]
\KeywordTok{$email} \NormalTok{= }\KeywordTok{$_POST}\OtherTok{[}\StringTok{'email'}\OtherTok{];}
\KeywordTok{if}\OtherTok{(}\FunctionTok{preg_match}\OtherTok{(}\StringTok{"~([a-zA-Z0-9!#$%&amp;'*+-/=?^_`\{|\}~])@([a-zA-Z0-9-]).([a-zA-Z0-9]\{2,4\})~"}\OtherTok{,}\KeywordTok{$email}\OtherTok{))} \NormalTok{\{}
    \FunctionTok{echo} \StringTok{'This is a valid email.'}\OtherTok{;}
\NormalTok{\} }\KeywordTok{else}\NormalTok{\{}
    \FunctionTok{echo} \StringTok{'This is an invalid email.'}\OtherTok{;}
\NormalTok{\} }
\end{Highlighting}
\end{Shaded}

\section{XML}\label{xml}

\begin{Shaded}
\begin{Highlighting}[]
\CommentTok{//this is a sample xml string}
\KeywordTok{$xml_string}\NormalTok{=}\StringTok{"<?xml version='1.0'?>}
\StringTok{<moleculedb>}
\StringTok{    <molecule name='Benzine'>}
\StringTok{        <symbol>ben</symbol>}
\StringTok{        <code>A</code>}
\StringTok{    </molecule>}
\StringTok{    <molecule name='Water'>}
\StringTok{        <symbol>h2o</symbol>}
\StringTok{        <code>K</code>}
\StringTok{    </molecule>}
\StringTok{</moleculedb>"}\OtherTok{;}

\CommentTok{//load the xml string using simplexml function}
\KeywordTok{$xml} \NormalTok{= }\FunctionTok{simplexml_load_string}\OtherTok{(}\KeywordTok{$xml_string}\OtherTok{);}

\CommentTok{//loop through the each node of molecule}
\KeywordTok{foreach} \OtherTok{(}\KeywordTok{$xml}\NormalTok{->molecule }\KeywordTok{as} \KeywordTok{$record}\OtherTok{)}
\NormalTok{\{}
   \CommentTok{//attribute are accessted by}
   \FunctionTok{echo} \KeywordTok{$record}\OtherTok{[}\StringTok{'name'}\OtherTok{],} \StringTok{'  '}\OtherTok{;}
   \CommentTok{//node are accessted by -> operator}
   \FunctionTok{echo} \KeywordTok{$record}\NormalTok{->symbol}\OtherTok{,} \StringTok{'  '}\OtherTok{;}
   \FunctionTok{echo} \KeywordTok{$record}\NormalTok{->code}\OtherTok{,} \StringTok{'<br />'}\OtherTok{;}
\NormalTok{\}}
\end{Highlighting}
\end{Shaded}

per leggere invece un file:

\begin{Shaded}
\begin{Highlighting}[]
\FunctionTok{simplexml_load_file}\OtherTok{(}\KeywordTok{$xml_file}\OtherTok{);}
\end{Highlighting}
\end{Shaded}

\section{JSON}\label{json}

Following is the PHP code to create the JSON data format of above
example using array of PHP.

\begin{Shaded}
\begin{Highlighting}[]
\KeywordTok{$json_data} \NormalTok{= }\FunctionTok{array} \OtherTok{(}\StringTok{'id'}\NormalTok{=>}\DecValTok{1}\OtherTok{,}\StringTok{'name'}\NormalTok{=>}\StringTok{"rolf"}\OtherTok{,}\StringTok{'country'}\NormalTok{=>}\StringTok{'russia'}\OtherTok{,}\StringTok{"office"}\NormalTok{=>}\FunctionTok{array}\OtherTok{(}\StringTok{"google"}\OtherTok{,}\StringTok{"oracle"}\OtherTok{));}
\FunctionTok{echo} \FunctionTok{json_encode}\OtherTok{(}\KeywordTok{$json_data}\OtherTok{);}
\end{Highlighting}
\end{Shaded}

Following code will parse the JSON data into PHP arrays.

\begin{Shaded}
\begin{Highlighting}[]
\KeywordTok{$json_string}\NormalTok{=}\StringTok{'\{"id":1,"name":"rolf","country":"russia","office":["google","oracle"]\} '}\OtherTok{;}
\KeywordTok{$obj}\NormalTok{=}\FunctionTok{json_decode}\OtherTok{(}\KeywordTok{$json_string}\OtherTok{);}
\CommentTok{//print the parsed data}
\FunctionTok{echo} \KeywordTok{$obj}\NormalTok{->name}\OtherTok{;} \CommentTok{//displays rolf}
\FunctionTok{echo} \KeywordTok{$obj}\NormalTok{->office}\OtherTok{[}\DecValTok{0}\OtherTok{];} \CommentTok{//displays google}
\end{Highlighting}
\end{Shaded}

\section{Varie}\label{varie}

Per vedere tutti i parametri passati con un POST

\begin{Shaded}
\begin{Highlighting}[]
\FunctionTok{echo} \StringTok{"Values submitted via POST method:<br />}\KeywordTok{\textbackslash{}n}\StringTok{"}\OtherTok{;}
\FunctionTok{reset} \OtherTok{(}\KeywordTok{$_POST}\OtherTok{);}
\KeywordTok{while} \OtherTok{(}\FunctionTok{list} \OtherTok{(}\KeywordTok{$key}\OtherTok{,} \KeywordTok{$val}\OtherTok{)} \NormalTok{= }\FunctionTok{each} \OtherTok{(}\KeywordTok{$_POST}\OtherTok{))} \NormalTok{\{}
    \FunctionTok{echo} \StringTok{"}\KeywordTok{$key}\StringTok{ => }\KeywordTok{$val}\StringTok{<br />}\KeywordTok{\textbackslash{}n}\StringTok{"}\OtherTok{;}
\NormalTok{\}}
\end{Highlighting}
\end{Shaded}

le variabili da POST GET:

\begin{verbatim}
$_GET["ID"]
\end{verbatim}

per eliminare I warnings dall'html: in php.ini:

\begin{verbatim}
error_reporting = E_ALL & ~E_notice & ~E_warnings
service httpd restart
\end{verbatim}

Alternativo IF

\begin{Shaded}
\begin{Highlighting}[]
\KeywordTok{<?php}
\KeywordTok{if} \OtherTok{(}\KeywordTok{$a} \NormalTok{> }\DecValTok{5}\OtherTok{)} \NormalTok{\{}
   \FunctionTok{echo} \StringTok{"big"}\OtherTok{;}
\NormalTok{\} }\KeywordTok{else} \NormalTok{\{}
   \FunctionTok{echo} \StringTok{"small"}\OtherTok{;}
\NormalTok{\}}
\KeywordTok{?>}
\NormalTok{can be replaced by:}
\NormalTok{<}\OtherTok{?}\NormalTok{php}
\FunctionTok{echo} \OtherTok{(}\KeywordTok{$a} \NormalTok{> }\DecValTok{5} \OtherTok{?} \StringTok{"big"} \OtherTok{:} \StringTok{"small"}\OtherTok{);}
\KeywordTok{?>}
\end{Highlighting}
\end{Shaded}

\section{Stringhe}\label{stringhe}

\begin{Shaded}
\begin{Highlighting}[]
\KeywordTok{$out} \NormalTok{= }\KeywordTok{<<<EOF}
\StringTok{<a href="ordini.php">visualizza rodini</a>}
\StringTok{<a href="amm.php">gestione articoli</a>}
\KeywordTok{EOF;}
\end{Highlighting}
\end{Shaded}

\end{document}
