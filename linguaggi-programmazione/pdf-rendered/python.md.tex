\documentclass[]{article}
\usepackage{lmodern}
\usepackage{amssymb,amsmath}
\usepackage{ifxetex,ifluatex}
\usepackage{fixltx2e} % provides \textsubscript
\ifnum 0\ifxetex 1\fi\ifluatex 1\fi=0 % if pdftex
  \usepackage[T1]{fontenc}
  \usepackage[utf8]{inputenc}
\else % if luatex or xelatex
  \ifxetex
    \usepackage{mathspec}
    \usepackage{xltxtra,xunicode}
  \else
    \usepackage{fontspec}
  \fi
  \defaultfontfeatures{Mapping=tex-text,Scale=MatchLowercase}
  \newcommand{\euro}{€}
\fi
% use upquote if available, for straight quotes in verbatim environments
\IfFileExists{upquote.sty}{\usepackage{upquote}}{}
% use microtype if available
\IfFileExists{microtype.sty}{\usepackage{microtype}}{}
\usepackage[top=1cm, bottom=1.5cm, left=1cm, right=1cm]{geometry}
\usepackage{color}
\usepackage{fancyvrb}
\newcommand{\VerbBar}{|}
\newcommand{\VERB}{\Verb[commandchars=\\\{\}]}
\DefineVerbatimEnvironment{Highlighting}{Verbatim}{commandchars=\\\{\}}
% Add ',fontsize=\small' for more characters per line
\newenvironment{Shaded}{}{}
\newcommand{\KeywordTok}[1]{\textcolor[rgb]{0.00,0.44,0.13}{\textbf{{#1}}}}
\newcommand{\DataTypeTok}[1]{\textcolor[rgb]{0.56,0.13,0.00}{{#1}}}
\newcommand{\DecValTok}[1]{\textcolor[rgb]{0.25,0.63,0.44}{{#1}}}
\newcommand{\BaseNTok}[1]{\textcolor[rgb]{0.25,0.63,0.44}{{#1}}}
\newcommand{\FloatTok}[1]{\textcolor[rgb]{0.25,0.63,0.44}{{#1}}}
\newcommand{\CharTok}[1]{\textcolor[rgb]{0.25,0.44,0.63}{{#1}}}
\newcommand{\StringTok}[1]{\textcolor[rgb]{0.25,0.44,0.63}{{#1}}}
\newcommand{\CommentTok}[1]{\textcolor[rgb]{0.38,0.63,0.69}{\textit{{#1}}}}
\newcommand{\OtherTok}[1]{\textcolor[rgb]{0.00,0.44,0.13}{{#1}}}
\newcommand{\AlertTok}[1]{\textcolor[rgb]{1.00,0.00,0.00}{\textbf{{#1}}}}
\newcommand{\FunctionTok}[1]{\textcolor[rgb]{0.02,0.16,0.49}{{#1}}}
\newcommand{\RegionMarkerTok}[1]{{#1}}
\newcommand{\ErrorTok}[1]{\textcolor[rgb]{1.00,0.00,0.00}{\textbf{{#1}}}}
\newcommand{\NormalTok}[1]{{#1}}
\usepackage{longtable,booktabs}
\ifxetex
  \usepackage[setpagesize=false, % page size defined by xetex
              unicode=false, % unicode breaks when used with xetex
              xetex]{hyperref}
\else
  \usepackage[unicode=true]{hyperref}
\fi
\hypersetup{breaklinks=true,
            bookmarks=true,
            pdfauthor={Giuliano Dedda},
            pdftitle={Python},
            colorlinks=true,
            citecolor=blue,
            urlcolor=blue,
            linkcolor=magenta,
            pdfborder={0 0 0}}
\urlstyle{same}  % don't use monospace font for urls
\setlength{\parindent}{0pt}
\setlength{\parskip}{6pt plus 2pt minus 1pt}
\setlength{\emergencystretch}{3em}  % prevent overfull lines
\setcounter{secnumdepth}{0}

\title{Python}
\author{Giuliano Dedda}
\date{14/07/2014}

\begin{document}
\maketitle

\section{Files}\label{files}

\subsection{Leggere e scrivere su
files}\label{leggere-e-scrivere-su-files}

\begin{Shaded}
\begin{Highlighting}[]
\NormalTok{output = }\DataTypeTok{open}\NormalTok{(}\StringTok{'pippo.txt'}\NormalTok{,}\StringTok{'w'}\NormalTok{)  apertura di un }\DataTypeTok{file} \NormalTok{in scrittura}
\DataTypeTok{input} \NormalTok{= }\DataTypeTok{open}\NormalTok{(}\StringTok{'dati'}\NormalTok{,}\StringTok{'r'}\NormalTok{)    apertura di un }\DataTypeTok{file} \NormalTok{in lettura}
\NormalTok{s = }\DataTypeTok{input}\NormalTok{.read()    lettura dell}\StringTok{'intero contenuto del file}
\StringTok{s = input.read(N)   lettura di N bytes}
\StringTok{s = input.readline()    lettura di una riga (per files di testo)}
\StringTok{s = input.readlines()   restuisce l'}\NormalTok{intero }\DataTypeTok{file} \NormalTok{come lista di righe (per files di testo)}
\NormalTok{output.wre(s)   scrivo un intero }\DataTypeTok{file}
\NormalTok{output.wrelines(L)  scrive la lista L in righe nel }\DataTypeTok{file}
\NormalTok{output.close(L) chiusura }\KeywordTok{del} \DataTypeTok{file}
\end{Highlighting}
\end{Shaded}

è meglio usare readlines rispetto a readline perché è più veloce.

\subsection{Parsing di un file}\label{parsing-di-un-file}

\begin{Shaded}
\begin{Highlighting}[]
\CharTok{import} \NormalTok{sys}

\NormalTok{f = }\DataTypeTok{open}\NormalTok{(sys.argv[}\DecValTok{1}\NormalTok{], }\StringTok{'r'}\NormalTok{)}

\KeywordTok{for} \NormalTok{line in f.readlines():}
    \DataTypeTok{print} \NormalTok{line}
\end{Highlighting}
\end{Shaded}

\subsection{File in una directory}\label{file-in-una-directory}

\begin{Shaded}
\begin{Highlighting}[]
\NormalTok{dirname = }\StringTok{'./'}
\KeywordTok{for} \NormalTok{dirpath, dirnames,files in os.walk(dirname):}
    \KeywordTok{for} \NormalTok{filename in files:}
        \NormalTok{file_full_path = os.path.join(dirpath, filename)}
        \DataTypeTok{print}\NormalTok{(file_full_path)}
\end{Highlighting}
\end{Shaded}

\subsection{Movimentazione file}\label{movimentazione-file}

\begin{Shaded}
\begin{Highlighting}[]
\NormalTok{(root,ext) = os.path.splitext(filename) }\CommentTok{# divide il nomefile in basename e estensione}

\NormalTok{shutil.copyfile(src,dst) }\CommentTok{# copia file}
\NormalTok{shutil.move(src,dst) }\CommentTok{# sposta file}
\end{Highlighting}
\end{Shaded}

\section{Script che mostra gli argomenti e esegue un
comando}\label{script-che-mostra-gli-argomenti-e-esegue-un-comando}

\begin{Shaded}
\begin{Highlighting}[]
\CharTok{import} \NormalTok{sys}
\CharTok{import} \NormalTok{os}

\DataTypeTok{print} \StringTok{'Begining... now...'}

\KeywordTok{for} \NormalTok{x in sys.argv:}
    \DataTypeTok{print} \NormalTok{x}

\NormalTok{os.system(}\StringTok{"echo ciao"}\NormalTok{)}
\end{Highlighting}
\end{Shaded}

\section{Array, liste, dizionari}\label{array-liste-dizionari}

Una Lista (cioè un array) può essere definita così:

\begin{Shaded}
\begin{Highlighting}[]
\NormalTok{xx = array([}\DecValTok{2}\NormalTok{, }\DecValTok{4}\NormalTok{, -}\DecValTok{11}\NormalTok{])}
\NormalTok{yy = zeros(}\DecValTok{3}\NormalTok{, Int)        }\CommentTok{# Create empty array ready to receive result}
\KeywordTok{for} \NormalTok{i in }\DataTypeTok{range}\NormalTok{(}\DecValTok{0}\NormalTok{, }\DecValTok{3}\NormalTok{):}
     \NormalTok{yy[i] = xx[i] * }\DecValTok{2}
\DataTypeTok{print} \NormalTok{yy}
\NormalTok{[  }\DecValTok{4}   \DecValTok{8} \NormalTok{-}\DecValTok{22}\NormalTok{]}

\NormalTok{li = [}\StringTok{"a"}\NormalTok{, }\StringTok{"b"}\NormalTok{, }\StringTok{"mpilgrim"}\NormalTok{, }\StringTok{"z"}\NormalTok{, }\StringTok{"example"}\NormalTok{]}


\NormalTok{Le }\DataTypeTok{tuple} \NormalTok{sono come le liste ma non hanno metodi e sono pià veloci}
    \NormalTok{t = (}\StringTok{"a"}\NormalTok{, }\StringTok{"b"}\NormalTok{, }\StringTok{"mpilgrim"}\NormalTok{, }\StringTok{"z"}\NormalTok{, }\StringTok{"example"}\NormalTok{)}


\NormalTok{Un dizionario invece ha una relazione uno a uno con le chiavi:}
    \NormalTok{d = \{}\StringTok{"server"}\NormalTok{:}\StringTok{"mpilgrim"}\NormalTok{, }\StringTok{"database"}\NormalTok{:}\StringTok{"master"}\NormalTok{\} }
    \DataTypeTok{print} \NormalTok{d[}\StringTok{'server'}\NormalTok{]}

\NormalTok{params = \{}\StringTok{"server"}\NormalTok{:}\StringTok{"mpilgrim"}\NormalTok{, }\StringTok{"database"}\NormalTok{:}\StringTok{"master"}\NormalTok{, }\StringTok{"uid"}\NormalTok{:}\StringTok{"sa"}\NormalTok{, }\StringTok{"pwd"}\NormalTok{:}\StringTok{"secret"}\NormalTok{\}}

\NormalTok{>>> params.keys()   }
\NormalTok{[}\StringTok{'server'}\NormalTok{, }\StringTok{'uid'}\NormalTok{, }\StringTok{'database'}\NormalTok{, }\StringTok{'pwd'}\NormalTok{]}

\NormalTok{>>> params.values() }
\NormalTok{[}\StringTok{'mpilgrim'}\NormalTok{, }\StringTok{'sa'}\NormalTok{, }\StringTok{'master'}\NormalTok{, }\StringTok{'secret'}\NormalTok{]}

\NormalTok{>>> params.items()  }
\NormalTok{[(}\StringTok{'server'}\NormalTok{, }\StringTok{'mpilgrim'}\NormalTok{), (}\StringTok{'uid'}\NormalTok{, }\StringTok{'sa'}\NormalTok{), (}\StringTok{'database'}\NormalTok{, }\StringTok{'master'}\NormalTok{), (}\StringTok{'pwd'}\NormalTok{, }\StringTok{'secret'}\NormalTok{)]}
\end{Highlighting}
\end{Shaded}

\section{Stringhe}\label{stringhe}

Per concatenare stringhe sui usa il `+'

\begin{longtable}[l]{@{}ll@{}}
\toprule\addlinespace
Comando & Effetto
\\\addlinespace
\midrule\endhead
print ``Users connected: \%d'' \% (userCount, ) & Inserire un intero in
una stringa
\\\addlinespace
stringa = format(int(data\_video{[}1{]}), `02d') & per formattare gli
interi con gli zero davanti (leading zero)
\\\addlinespace
testo = testo.replace(`S', `6') & sostituzione di caratteri all'interno
di una stringa
\\\addlinespace
\bottomrule
\end{longtable}

\section{Espressioni regolari}\label{espressioni-regolari}

\begin{Shaded}
\begin{Highlighting}[]
\CharTok{import} \NormalTok{sys}
\CharTok{import} \NormalTok{re}

\NormalTok{NomeFile = sys.argv[}\DecValTok{1}\NormalTok{]}

\NormalTok{f = }\DataTypeTok{open}\NormalTok{(NomeFile, }\StringTok{'r'}\NormalTok{)}
\KeywordTok{for} \NormalTok{line in f.readlines():}
    \KeywordTok{if} \NormalTok{re.search(}\StringTok{'espessione'}\NormalTok{, line,re.I):}
        \DataTypeTok{print} \NormalTok{line}
\end{Highlighting}
\end{Shaded}

\subsection{Esempi di espressioni}\label{esempi-di-espressioni}

\begin{longtable}[l]{@{}ll@{}}
\toprule\addlinespace
Comando & Effetto
\\\addlinespace
\midrule\endhead
parola & Cerca la parola all'interno della stringa
\\\addlinespace
\^{}parola & Cerca la parola all'inizio della stringa
\\\addlinespace
parola\$ & Cerca la parola alla fine della stringa
\\\addlinespace
cas{[}ae{]} & Cerca la parola seguita dalle lettere nelle parentesi
quadre {[}{]} es trova sia `casa' che `case'
\\\addlinespace
\bottomrule
\end{longtable}

\section{Import}\label{import}

Per importare da librerie fatte da me

\begin{Shaded}
\begin{Highlighting}[]
\NormalTok{sys.path.append(}\StringTok{"~/bin/"}\NormalTok{)}
\CharTok{import} \NormalTok{dvd_extract_vob_chapter}
\end{Highlighting}
\end{Shaded}

\section{Sqlite}\label{sqlite}

\begin{Shaded}
\begin{Highlighting}[]
\NormalTok{conn = sqlite3.}\OtherTok{connect}\NormalTok{(}\StringTok{'photos.db'}\NormalTok{)}
\NormalTok{c = conn.cursor()}
\NormalTok{LineSQL = }\StringTok{"SELECT base_uri,filename FROM photos, photo_tags WHERE photos.id=photo_id AND tag_id=1;"}

\NormalTok{c.execute(LineSQL)}
\KeywordTok{for} \NormalTok{row in c:}
    \DataTypeTok{print} \NormalTok{row}
\end{Highlighting}
\end{Shaded}

\end{document}
