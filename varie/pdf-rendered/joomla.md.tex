\documentclass[]{article}
\usepackage{lmodern}
\usepackage{amssymb,amsmath}
\usepackage{ifxetex,ifluatex}
\usepackage{fixltx2e} % provides \textsubscript
\ifnum 0\ifxetex 1\fi\ifluatex 1\fi=0 % if pdftex
  \usepackage[T1]{fontenc}
  \usepackage[utf8]{inputenc}
\else % if luatex or xelatex
  \ifxetex
    \usepackage{mathspec}
    \usepackage{xltxtra,xunicode}
  \else
    \usepackage{fontspec}
  \fi
  \defaultfontfeatures{Mapping=tex-text,Scale=MatchLowercase}
  \newcommand{\euro}{€}
\fi
% use upquote if available, for straight quotes in verbatim environments
\IfFileExists{upquote.sty}{\usepackage{upquote}}{}
% use microtype if available
\IfFileExists{microtype.sty}{\usepackage{microtype}}{}
\usepackage[top=1cm, bottom=1.5cm, left=1cm, right=1cm]{geometry}
\usepackage{color}
\usepackage{fancyvrb}
\newcommand{\VerbBar}{|}
\newcommand{\VERB}{\Verb[commandchars=\\\{\}]}
\DefineVerbatimEnvironment{Highlighting}{Verbatim}{commandchars=\\\{\}}
% Add ',fontsize=\small' for more characters per line
\newenvironment{Shaded}{}{}
\newcommand{\KeywordTok}[1]{\textcolor[rgb]{0.00,0.44,0.13}{\textbf{{#1}}}}
\newcommand{\DataTypeTok}[1]{\textcolor[rgb]{0.56,0.13,0.00}{{#1}}}
\newcommand{\DecValTok}[1]{\textcolor[rgb]{0.25,0.63,0.44}{{#1}}}
\newcommand{\BaseNTok}[1]{\textcolor[rgb]{0.25,0.63,0.44}{{#1}}}
\newcommand{\FloatTok}[1]{\textcolor[rgb]{0.25,0.63,0.44}{{#1}}}
\newcommand{\CharTok}[1]{\textcolor[rgb]{0.25,0.44,0.63}{{#1}}}
\newcommand{\StringTok}[1]{\textcolor[rgb]{0.25,0.44,0.63}{{#1}}}
\newcommand{\CommentTok}[1]{\textcolor[rgb]{0.38,0.63,0.69}{\textit{{#1}}}}
\newcommand{\OtherTok}[1]{\textcolor[rgb]{0.00,0.44,0.13}{{#1}}}
\newcommand{\AlertTok}[1]{\textcolor[rgb]{1.00,0.00,0.00}{\textbf{{#1}}}}
\newcommand{\FunctionTok}[1]{\textcolor[rgb]{0.02,0.16,0.49}{{#1}}}
\newcommand{\RegionMarkerTok}[1]{{#1}}
\newcommand{\ErrorTok}[1]{\textcolor[rgb]{1.00,0.00,0.00}{\textbf{{#1}}}}
\newcommand{\NormalTok}[1]{{#1}}
\ifxetex
  \usepackage[setpagesize=false, % page size defined by xetex
              unicode=false, % unicode breaks when used with xetex
              xetex]{hyperref}
\else
  \usepackage[unicode=true]{hyperref}
\fi
\hypersetup{breaklinks=true,
            bookmarks=true,
            pdfauthor={},
            pdftitle={},
            colorlinks=true,
            citecolor=blue,
            urlcolor=blue,
            linkcolor=magenta,
            pdfborder={0 0 0}}
\urlstyle{same}  % don't use monospace font for urls
\setlength{\parindent}{0pt}
\setlength{\parskip}{6pt plus 2pt minus 1pt}
\setlength{\emergencystretch}{3em}  % prevent overfull lines
\setcounter{secnumdepth}{0}


\begin{document}

\section{JOOMLA}\label{joomla}

\section{Installazione}\label{installazione}

conviene dare i permessi alla cartella all'utente con cui gira il web
server es supponendo che joomla sia installata in
/home/test/public\_html/j02 e apache giri con utente: http

\begin{verbatim}
sudo chgrp -R http *
chmod g+w -R j02/
\end{verbatim}

\section{Creazioni componenti}\label{creazioni-componenti}

\subsection{visualizzazione nel menu}\label{visualizzazione-nel-menu}

per visualizzare il componente tra i menù disponibili creare il file
default.xml all'interno delle view da mostrare es:
site/views/helloworld/tmpl/default.xml\\NB: per
COM\_HELLOWORLD\_HELLOWORLD\_VIEW\_DEFAULT\_DESC utilizzare il file di
language dentro admin

\begin{Shaded}
\begin{Highlighting}[]
\KeywordTok{<?xml} \NormalTok{version="1.0" encoding="utf-8"}\KeywordTok{?>}
\KeywordTok{<metadata>}
        \KeywordTok{<layout}\OtherTok{ title=}\StringTok{"COM_HELLOWORLD_HELLOWORLD_VIEW_DEFAULT_TITLE"}\KeywordTok{>}
                \KeywordTok{<message>}
                        \BaseNTok{<![CDATA[}\NormalTok{COM_HELLOWORLD_HELLOWORLD_VIEW_DEFAULT_DESC}\BaseNTok{]]>}
                \KeywordTok{</message>}
        \KeywordTok{</layout>}
\KeywordTok{</metadata>}
\end{Highlighting}
\end{Shaded}

\subsection{Variabili utili}\label{variabili-utili}

\begin{Shaded}
\begin{Highlighting}[]
\KeywordTok{JPATH_ADMINISTRATOR}     \NormalTok{->  /home/test01/public_html/j01/administrator}
\KeywordTok{JPATH_ROOT}              \NormalTok{->  /home/test01/public_html/j01}
\KeywordTok{JPATH_COMPONENT}         \NormalTok{->  /home/test01/public_html/j01/components/com_pqz}
\KeywordTok{JPATH_LIBRARIES}         \NormalTok{->  /home/test01/public_html/j01/libraries}
\KeywordTok{JURI}\NormalTok{::root}\OtherTok{()}            \NormalTok{->  http:}\CommentTok{//192.168.1.101/~test01/j01/}
\NormalTok{Juri::base}\OtherTok{()}            \NormalTok{->  http:}\CommentTok{//192.168.20.27/~test01/j01/ }
\end{Highlighting}
\end{Shaded}

\subsubsection{Dati Utente}\label{dati-utente}

\begin{Shaded}
\begin{Highlighting}[]
\KeywordTok{$user} \NormalTok{= JFactory::getUser}\OtherTok{();}
 
\KeywordTok{if} \OtherTok{(}\NormalTok{!}\KeywordTok{$user}\NormalTok{->guest}\OtherTok{)} \NormalTok{\{}
  \FunctionTok{echo} \StringTok{'You are logged in as:<br />'}\OtherTok{;}
  \FunctionTok{echo} \StringTok{'User name: '} \NormalTok{. }\KeywordTok{$user}\NormalTok{->username . }\StringTok{'<br />'}\OtherTok{;}
  \FunctionTok{echo} \StringTok{'Real name: '} \NormalTok{. }\KeywordTok{$user}\NormalTok{->name . }\StringTok{'<br />'}\OtherTok{;}
  \FunctionTok{echo} \StringTok{'User ID  : '} \NormalTok{. }\KeywordTok{$user}\NormalTok{->id . }\StringTok{'<br />'}\OtherTok{;}
\NormalTok{\}}
\end{Highlighting}
\end{Shaded}

\subsubsection{Accesso al db}\label{accesso-al-db}

\begin{Shaded}
\begin{Highlighting}[]
\CommentTok{// Get a db connection.}
\KeywordTok{$db} \NormalTok{= JFactory::getDbo}\OtherTok{();}
 
\CommentTok{// Create a new query object.}
\KeywordTok{$query} \NormalTok{= }\KeywordTok{$db}\NormalTok{->getQuery}\OtherTok{(}\KeywordTok{true}\OtherTok{);}
 
\CommentTok{// Select all records from the user profile table where key begins with "custom.".}
\CommentTok{// Order it by the ordering field.}
\KeywordTok{$query}\NormalTok{->select}\OtherTok{(}\KeywordTok{$db}\NormalTok{->quoteName}\OtherTok{(}\FunctionTok{array}\OtherTok{(}\StringTok{'user_id'}\OtherTok{,} \StringTok{'profile_key'}\OtherTok{,} \StringTok{'profile_value'}\OtherTok{,} \StringTok{'ordering'}\OtherTok{)));}
\KeywordTok{$query}\NormalTok{->from}\OtherTok{(}\KeywordTok{$db}\NormalTok{->quoteName}\OtherTok{(}\StringTok{'#__user_profiles'}\OtherTok{));}
\KeywordTok{$query}\NormalTok{->where}\OtherTok{(}\KeywordTok{$db}\NormalTok{->quoteName}\OtherTok{(}\StringTok{'profile_key'}\OtherTok{)} \NormalTok{. }\StringTok{' LIKE '}\NormalTok{. }\KeywordTok{$db}\NormalTok{->quote}\OtherTok{(}\StringTok{'}\KeywordTok{\textbackslash{}'}\StringTok{custom.%}\KeywordTok{\textbackslash{}'}\StringTok{'}\OtherTok{));}
\KeywordTok{$query}\NormalTok{->order}\OtherTok{(}\StringTok{'ordering ASC'}\OtherTok{);}
 
\CommentTok{// Reset the query using our newly populated query object.}
\KeywordTok{$db}\NormalTok{->setQuery}\OtherTok{(}\KeywordTok{$query}\OtherTok{);}
 
\CommentTok{// Load the results as a list of stdClass objects (see later for more options on retrieving data).}
\KeywordTok{$results} \NormalTok{= }\KeywordTok{$db}\NormalTok{->loadObjectList}\OtherTok{();}
\end{Highlighting}
\end{Shaded}

comunque guardare:
http://docs.joomla.org/Selecting\_data\_using\_JDatabase

\subsection{HTML}\label{html}

JHtml::\_(`grid.checkall') -\textgreater{} mostra il chackbox `seleziona
tutti' (da capire come usarlo in tabella però)

per inserire i css del modulo, metterlo nella view (default.php)

\begin{Shaded}
\begin{Highlighting}[]
\NormalTok{JHtml::stylesheet}\OtherTok{(}\NormalTok{Juri::base}\OtherTok{()} \NormalTok{. }\StringTok{'components/com_pqz/media/css/com_pqz.css'}\OtherTok{);}
\end{Highlighting}
\end{Shaded}

\subsection{Parametri in input}\label{parametri-in-input}

\begin{Shaded}
\begin{Highlighting}[]
\KeywordTok{$jinput} \NormalTok{= JFactory::getApplication}\OtherTok{()}\NormalTok{->input}\OtherTok{;}
\KeywordTok{$foo} \NormalTok{= }\KeywordTok{$jinput}\NormalTok{->get}\OtherTok{(}\StringTok{'varname'}\OtherTok{,} \StringTok{'default_value'}\OtherTok{,} \StringTok{'filter'}\OtherTok{);}
\end{Highlighting}
\end{Shaded}

per i filtri possibili vedere:
http://docs.joomla.org/Retrieving\_request\_data\_using\_JInput

\subsection{Internazionalizzazione}\label{internazionalizzazione}

JText::\emph{(`COM\_PQZ\_TEST') -\textgreater{} mostra il testo nella
linga corrispondente relativo a COM}PQZ\_TEST (es in
language/en-GB/en-GB.com\_pqz.ini)
JText::sprintf(`COM\_PQZ\_TEST',$stringa1,$stringa2) -\textgreater{}
sustituisce i \%s nella stringa COM\_PQZ\_TEST con i valori delle
stringhe

\section{MCV}\label{mcv}

per passare da una view (ad esempio se aggianciata direttamente da un
menù con il file default.xml) bisogna creare un modello con il nome:
della view ed all'interno la classe nome\_compnenteModelNomevista. es:
file: view/edit\_csv/view.html.php

\begin{Shaded}
\begin{Highlighting}[]
\KeywordTok{class} \NormalTok{pqzViewedit_csv }\KeywordTok{extends} \NormalTok{JViewLegacy \{}
\end{Highlighting}
\end{Shaded}

file models/edit\_csv.php

\begin{Shaded}
\begin{Highlighting}[]
\KeywordTok{class} \NormalTok{pqzModeledit_csv }\KeywordTok{extends} \NormalTok{JModelList \{}
\end{Highlighting}
\end{Shaded}

\subsection{JViewLegacy}\label{jviewlegacy}

\begin{Shaded}
\begin{Highlighting}[]
\KeywordTok{$this}\NormalTok{->getNmae}\OtherTok{();} \CommentTok{// print the name of the view}
\end{Highlighting}
\end{Shaded}

\subsection{Controller}\label{controller}

la funzione di default si chiama `default' il model di default ha la lo
stesso nome del componente, ma si può richiamare un model diverso (es
helloworld\_model.php) con la funzione:
$model = $this-\textgreater{}getModel(`helloworld\_model'); poi si
possono anche lanciare le funzioni direttamente, ma è meglio impostare
anche il modello della view es:

\begin{Shaded}
\begin{Highlighting}[]
\KeywordTok{$model} \NormalTok{= }\KeywordTok{$this}\NormalTok{->getModel}\OtherTok{(}\StringTok{'choose_quiz'}\OtherTok{);}
\KeywordTok{$view} \NormalTok{= }\KeywordTok{$this}\NormalTok{->getView}\OtherTok{(}\StringTok{'choose_quiz'}\OtherTok{,} \StringTok{'html'}\OtherTok{);}
\KeywordTok{$view}\NormalTok{->setModel}\OtherTok{(} \KeywordTok{$model} \OtherTok{,} \KeywordTok{true} \OtherTok{);}
\end{Highlighting}
\end{Shaded}

\subsection{Models}\label{models}

il modello chiamato (es choose\_quiz.pgp) ha una funzione public
function getItems() \{

\subsection{View}\label{view}

per ottenere i dati dal modello (funzione Get Items). (il modello
dovrebbe essere stato settato dal controller

\begin{Shaded}
\begin{Highlighting}[]
\KeywordTok{$out} \NormalTok{= }\KeywordTok{$this}\NormalTok{->get}\OtherTok{(}\StringTok{'Items'}\OtherTok{);}
\NormalTok{print_pre}\OtherTok{(}\KeywordTok{$out}\OtherTok{);}
\end{Highlighting}
\end{Shaded}

è possibile avere più template nella stessa view ad esempio per chiamare
tmpl/default\_question.php si usa

\begin{Shaded}
\begin{Highlighting}[]
\FunctionTok{echo} \KeywordTok{$this}\NormalTok{->loadTemplate}\OtherTok{(}\StringTok{"question"}\OtherTok{);}
\end{Highlighting}
\end{Shaded}

\end{document}
