\documentclass[a4paper]{article}

\usepackage{amsmath,amssymb}   % per le formule matematiche

\begin{document}

\section{capitolo 1}
Prova latex
\section{formule Matematiche}

\subsection{formule in corpo}
Ci sono voluti secoli per
dimostrare che quando $n>2$
\emph{non} ci sono tre interi
positivi $a$, $b$, $c$ tali che $
a^n+b^n=c^n$.

\subsection{formule fuori corpo}

Se $f$ è continua e
\[
F(x)=\int_a^x f(t)\,dt,
\]

(equzaione senza indice)

allora
\begin{equation}
F'(x)=f(x).
\end{equation}


\begin{equation}
	\label{eq:euler}
		e^{i\pi}+1=0.
	\end{equation}

Dalla formula~\eqref{eq:euler} si deduce che\dots

(equzaione con indice)

\end{document}
