\documentclass[]{article}
\usepackage{lmodern}
\usepackage{amssymb,amsmath}
\usepackage{ifxetex,ifluatex}
\usepackage{fixltx2e} % provides \textsubscript
\ifnum 0\ifxetex 1\fi\ifluatex 1\fi=0 % if pdftex
  \usepackage[T1]{fontenc}
  \usepackage[utf8]{inputenc}
\else % if luatex or xelatex
  \ifxetex
    \usepackage{mathspec}
    \usepackage{xltxtra,xunicode}
  \else
    \usepackage{fontspec}
  \fi
  \defaultfontfeatures{Mapping=tex-text,Scale=MatchLowercase}
  \newcommand{\euro}{€}
\fi
% use upquote if available, for straight quotes in verbatim environments
\IfFileExists{upquote.sty}{\usepackage{upquote}}{}
% use microtype if available
\IfFileExists{microtype.sty}{\usepackage{microtype}}{}
\usepackage[top=1cm, bottom=1.5cm, left=1cm, right=1cm]{geometry}
\ifxetex
  \usepackage[setpagesize=false, % page size defined by xetex
              unicode=false, % unicode breaks when used with xetex
              xetex]{hyperref}
\else
  \usepackage[unicode=true]{hyperref}
\fi
\hypersetup{breaklinks=true,
            bookmarks=true,
            pdfauthor={Giuliano Dedda},
            pdftitle={Apache e SSL},
            colorlinks=true,
            citecolor=blue,
            urlcolor=blue,
            linkcolor=magenta,
            pdfborder={0 0 0}}
\urlstyle{same}  % don't use monospace font for urls
\setlength{\parindent}{0pt}
\setlength{\parskip}{6pt plus 2pt minus 1pt}
\setlength{\emergencystretch}{3em}  % prevent overfull lines
\setcounter{secnumdepth}{0}

\title{Apache e SSL}
\author{Giuliano Dedda}
\date{18/07/2014}

\begin{document}
\maketitle

\section{VirtualHost}\label{virtualhost}

Prima di fare alti virtualhost è meglio spostare quello di default in
/var/www/default

\begin{verbatim}
mkdir /var/www/nuovosito
mkdir /var/www/nuovosito/httpdocs
echo "funziona" > /var/www/nuovosito/httpdocs/index.html
\end{verbatim}

vi /etc/apache2/sites-available/001-www.nuovosito.it.conf

\begin{verbatim}
<VirtualHost *:80>
ServerName www.nuovosito.it
DocumentRoot /var/www/nuovosito/httpdocs
</VirtualHost>
\end{verbatim}

\section{Userdir}\label{userdir}

\begin{verbatim}
a2enmod userdir
\end{verbatim}

abilita la directory nelle home public\_html

per abilitare anche il php andare in /etc/apache2/mods-enabled/php5.conf

\begin{verbatim}
<IfModule mod_userdir.c>
    <Directory /home/*/public_html>
       php_admin_value engine Off
    </Directory>
</IfModule>
\end{verbatim}

e commentarle con un \#

\end{document}
