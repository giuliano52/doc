\documentclass[]{article}
\usepackage{lmodern}
\usepackage{amssymb,amsmath}
\usepackage{ifxetex,ifluatex}
\usepackage{fixltx2e} % provides \textsubscript
\ifnum 0\ifxetex 1\fi\ifluatex 1\fi=0 % if pdftex
  \usepackage[T1]{fontenc}
  \usepackage[utf8]{inputenc}
\else % if luatex or xelatex
  \ifxetex
    \usepackage{mathspec}
    \usepackage{xltxtra,xunicode}
  \else
    \usepackage{fontspec}
  \fi
  \defaultfontfeatures{Mapping=tex-text,Scale=MatchLowercase}
  \newcommand{\euro}{€}
\fi
% use upquote if available, for straight quotes in verbatim environments
\IfFileExists{upquote.sty}{\usepackage{upquote}}{}
% use microtype if available
\IfFileExists{microtype.sty}{\usepackage{microtype}}{}
\usepackage[top=1cm, bottom=1.5cm, left=1cm, right=1cm]{geometry}
\usepackage{longtable,booktabs}
\ifxetex
  \usepackage[setpagesize=false, % page size defined by xetex
              unicode=false, % unicode breaks when used with xetex
              xetex]{hyperref}
\else
  \usepackage[unicode=true]{hyperref}
\fi
\hypersetup{breaklinks=true,
            bookmarks=true,
            pdfauthor={Giuliano Dedda},
            pdftitle={Comandi Utili},
            colorlinks=true,
            citecolor=blue,
            urlcolor=blue,
            linkcolor=magenta,
            pdfborder={0 0 0}}
\urlstyle{same}  % don't use monospace font for urls
\setlength{\parindent}{0pt}
\setlength{\parskip}{6pt plus 2pt minus 1pt}
\setlength{\emergencystretch}{3em}  % prevent overfull lines
\setcounter{secnumdepth}{0}

\title{Comandi Utili}
\author{Giuliano Dedda}
\date{10/07/2014}

\begin{document}
\maketitle

\section{Operazioni su files}\label{operazioni-su-files}

\begin{longtable}[l]{@{}ll@{}}
\toprule\addlinespace
Comando & Effetto
\\\addlinespace
\midrule\endhead
sed -i `s/orig/dst/' file.txt & Sostituisce il testo orig con dst nel
file.txt
\\\addlinespace
sed `s/{[}\^{}{[}:print:{]}{]}/\#/g' input.log \textgreater{} output.txt
& Rimuove tutti i caratteri non stampabili
\\\addlinespace
sort in.txt \textgreater{} output.txt & Ordina un file di testo
\\\addlinespace
cat file1.txt file2.txt file*.txt \textgreater{} output.txt & unire più
files di testo in uno solo
\\\addlinespace
\bottomrule
\end{longtable}

\section{Filesystem}\label{filesystem}

\begin{longtable}[l]{@{}ll@{}}
\toprule\addlinespace
Comando & Effetto
\\\addlinespace
\midrule\endhead
grep parola * & Cerca una parola all'interno dei file -i (case
insensitive) -v (escludi parola)
\\\addlinespace
lndir /dir/sorgente /dir/destinazione & crea una copia di link di una
directory (usare i nomi delle directory completi)
\\\addlinespace
lsof & mostra i files aperti
\\\addlinespace
parted & serve per ridimensionare le partizioni
\\\addlinespace
mmls & Gestisce i dischi creati con dd
\\\addlinespace
fdupes -r ./ & Trova i files duplicati in una directory
\\\addlinespace
\bottomrule
\end{longtable}

\section{Comando Find}\label{comando-find}

Esegue una funzione dopo il find

\begin{verbatim}
find ./ -name '*.pdf' -exec ls -la {} \;     
find ./ -name '*.c' -exec grep ciao {} \;
find ./ -name '*.pdf' -exec rm -f {} \;
\end{verbatim}

copia una directory completamente in un altra

\begin{verbatim}
find . -depth -print0 | cpio --null --sparse --preserve-modification-time -pvd /mnt/newhome/
find /SouceDir -xdev | cpio -pm /DestDir`  
\end{verbatim}

Trova le directory vuote

\begin{verbatim}
find . -type d -empty
\end{verbatim}

Trova e cancella le directory vuote

\begin{verbatim}
find . -type d -empty -exec rmdir {} \;
\end{verbatim}

\section{Stato sistema}\label{stato-sistema}

\begin{longtable}[l]{@{}ll@{}}
\toprule\addlinespace
Comando & Effetto
\\\addlinespace
\midrule\endhead
free & Informazioni sulla memoria (La memoria utilizzata è il primo
numero della riga: -/+ buffers/cache: )
\\\addlinespace
lshw lshw -html & Mostra l'hardware presente sulla macchina
(eventualmente in formato html)
\\\addlinespace
lsusb & Mostra le periferiche usb (i dischi vengono visti come dischi
scsi)
\\\addlinespace
uptime & Mostra le informazioni di uptime, numero di utenti e carico
\\\addlinespace
id & Mostra gli id dell'utente
\\\addlinespace
lsblk & Mostra i dischi e le partizioni
\\\addlinespace
\bottomrule
\end{longtable}

\end{document}
