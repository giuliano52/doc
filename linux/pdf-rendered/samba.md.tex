\documentclass[]{article}
\usepackage{lmodern}
\usepackage{amssymb,amsmath}
\usepackage{ifxetex,ifluatex}
\usepackage{fixltx2e} % provides \textsubscript
\ifnum 0\ifxetex 1\fi\ifluatex 1\fi=0 % if pdftex
  \usepackage[T1]{fontenc}
  \usepackage[utf8]{inputenc}
\else % if luatex or xelatex
  \ifxetex
    \usepackage{mathspec}
    \usepackage{xltxtra,xunicode}
  \else
    \usepackage{fontspec}
  \fi
  \defaultfontfeatures{Mapping=tex-text,Scale=MatchLowercase}
  \newcommand{\euro}{€}
\fi
% use upquote if available, for straight quotes in verbatim environments
\IfFileExists{upquote.sty}{\usepackage{upquote}}{}
% use microtype if available
\IfFileExists{microtype.sty}{\usepackage{microtype}}{}
\usepackage[top=1cm, bottom=1.5cm, left=1cm, right=1cm]{geometry}
\ifxetex
  \usepackage[setpagesize=false, % page size defined by xetex
              unicode=false, % unicode breaks when used with xetex
              xetex]{hyperref}
\else
  \usepackage[unicode=true]{hyperref}
\fi
\hypersetup{breaklinks=true,
            bookmarks=true,
            pdfauthor={Giuliano Dedda},
            pdftitle={Samba},
            colorlinks=true,
            citecolor=blue,
            urlcolor=blue,
            linkcolor=magenta,
            pdfborder={0 0 0}}
\urlstyle{same}  % don't use monospace font for urls
\setlength{\parindent}{0pt}
\setlength{\parskip}{6pt plus 2pt minus 1pt}
\setlength{\emergencystretch}{3em}  % prevent overfull lines
\setcounter{secnumdepth}{0}

\title{Samba}
\author{Giuliano Dedda}
\date{20/07/2014}

\begin{document}
\maketitle

\section{Client}\label{client}

Per accedere alle share bisogna impostare il dominio. (Può essere
impostato anche in /etc/samba/smb.conf)

\begin{verbatim}
smbclient //host/share -U guest
smbclient -L //host -U guest
\end{verbatim}

se si vuole montare bisogna installare il pacchetto smbfs

\begin{verbatim}
smbmount //host/share /mnt/point/ -o username=guest
\end{verbatim}

In fsab

\begin{verbatim}
//mio_server_win2k/Cartella_condivisa
/directory_in_cui_monto_la_condivisione smbfs uid=<num_uid>,fmask=<fmask>,dmask=<dmask>,username=USER,password=PASS,workgroup=WORK 0 0
\end{verbatim}

\section{Server}\label{server}

\begin{verbatim}
 apt-get install samba 
 pacman -S samba
\end{verbatim}

vi /etc/samba/smb.conf

Per fare la condivisione delle home aggiungere alla fine di
/etc/samba/smb.conf: {[}homes{]} browseable = yes read only = no

\begin{verbatim}
sudo  smbpasswd -a username
\end{verbatim}

dalla versione 4 usare:

\begin{verbatim}
pdbedit -a -u <user>


systemctl enable smbd.service
systemctl enable nmbd.service
\end{verbatim}

accedere con

\begin{verbatim}
\\server-samba\username
\end{verbatim}

\section{Winbind}\label{winbind}

\begin{verbatim}
ntlm_auth –username=test01  | per verificare le credenziali dell'utente
wbinfo -n test01            | per trovare il sid di un utente
\end{verbatim}

\end{document}
